%!TEX root = ../article.tex

% Abstract
\begin{abstract}
Der Marching Cubes Algorithmus ist ein Algorithmus  zur Erzeugung von Oberflächenstrukturen aus Voxeldaten, der im medizinischen Bereich zur Rekonstruktion gemessener 3D-Daten verwendet wird. In diesem Paper wird eine Methode beschrieben, mit dem Marching Cubes Algorithmus, aus konstruierten Voxeldaten, Oberflächen zu rendern. Dazu wird eine Formel vorgestellt, anhand derer beim Erstellen der Marching Cubes die Vertices so verschoben werden, dass die Größe des gerenderten Ergebnisses in Relation zu den Füllmengen der Voxel steht. Auch wird gezeigt, wie unterschiedliche, den Voxeln zugeordnete Materialien auf dem gerenderten Ergebnis über Texturen dargestellt werden können.
\end{abstract}
