%!TEX root = ../article.tex

% Conclusion
\section{Implementierung der  Texturierung}
\label{sec:implementierungTexturierung}
Da bei einem Vertex, der zwischen zwei Voxeln liegt, die beide eine Füllmenge $>$ 0 haben, das Material beider Voxel beim Texturieren beachtet werden soll, kann nicht, wie in Abschnitt \ref{subsec:materialInterpolation} beschrieben, jedem Vertex ein primärer Voxel zugewiesen werden.

Die Mischverhältnisse der Materialien werden daher anhand der Position innerhalb des jeweiligen Achter-Tupels bestimmt. Dabei haben die Füllmengen der Voxel ebenfalls eine Auswirkung auf diese Mischverhältnisse.
\\

Der Abstand zu jedem Voxel im Achter-Tupel berechnet sich über folgende Formel (Dabei sind die acht Voxel innerhalb des Tupels über $i \in [0,7]$ nummeriert)
\[a_i= dif(x_i,x)*dif(y_i,y)*dif(z_i,z)\]
wobei $a_i$ = Abstand vom $i$ten Voxel, $(x,y,z)$ = Position innerhalb des Tupels, $(x_i,y_i,z_i)$ Position des $i$ten Voxels innerhalb des Tupels und $dif(a,b)$ = Differenz zwischen $a$ und $b$.
\\

Die Materialmischung berechnet sich über die Formel
\[m = \frac{\sum\limits_{i=0}^{7}m_i*a_i*f_i}{\sum\limits_{i=0}^{7}a_i*f_i}\]
wobei $m$ = Mischmaterial, $m_i$ = Materialeigenschaft (z.B.Farbe)  des $i$ten Voxels, $a_i$ = Abstand vom $i$ten Voxel und $f_i$ = Füllmenge des $i$ten Voxels.