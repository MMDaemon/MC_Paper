%!TEX root = ../article.tex

% Conclusion
\section{Einleitung}
\label{sec:einleitung}
Es gibt verschiedene Methoden Voxel zu rendern. Viele davon werden verwendet um konstruierte Datensätze zu rendern oder auch gemessene Daten zu repräsentieren. Dabei werden Voxel meist binär als gefüllt oder nicht gefüllt unterschieden und entweder mit einer uniformen Größe gerendert oder anhand der gemessenen Werte skaliert. Auch wird diesen oft ein Material zugewiesen, um Voxel unterschiedlich zu nutzen, wie beispielsweise dem Darstellen einer Landschaft oder dem Unterscheiden von Organen in einem medizinischen Scan.

In dieser Arbeit wird eine konstruktive Methode gezeigt Voxel zu rendern, die jeweils eine Füllmenge und ein Material zugewiesen haben. Diese Methode wurde auf Basis eines Projektes entwickelt. \cite{project}