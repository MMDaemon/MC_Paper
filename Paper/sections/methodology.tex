%!TEX root = ../article.tex

% Introduction
\section{Vorgehen}
Um mögliche Auswirkungen einer EDID-Manipulation reproduzierbar prüfen zu können, wird ein Testsystem benötigt. Hierfür wurde ein \emph{Raspberry Pi 3B} mit Raspbian als Betriebssystem sowie ein \emph{ATmega2560} Microcontroller (Elegoo MEGA 2560 R3) zusammen mit einem HDMI Breakout-Board eingesetzt. Damit lässt sich ein Monitor simulieren indem der Raspberry die EDID-Daten über den \interintegratedbus-Bus anfragt und von dem Microcontroller geliefert bekommt. 

Auf diese Weise besteht die Möglichkeit die EDID-Verarbeitung nicht nur direkt zu testen sondern, aufgrund der Quelloffenheit von Linux, auch direkt im Quelltext statisch zu analysieren.

\subsection{Versuchsaufbau}
Der Versuchsaufbau konzentriert sich auf Plug\&Play sowie die Übertragung von EDID-Blöcken über \interintegratedbus. 
Aus diesem Grund werden nicht alle HDMI Pins genutzt. Tabelle \ref{TblRelevantePins} stellt die relevanten Pins für diesen Aufbau dar. 
Wie sich dieser Versuchsaufbau erweitern ließe wird in \cref{sec:ausblick} skizziert.


\hspace*{-4cm}
\begin{table}[H]
	\caption{Genutzte Pins des HDMI Ports}
	\def\arraystretch{1.5}
	\centering
	\small
	\setlength\tabcolsep{1pt}
	\label{TblRelevantePins}
	\makebox[-1.1 \textwidth][c]{       %centering table
\begin{tabular}{|c|c|c|}
	\hline 
	\textbf{Pin} & \textbf{Funktion} & \textbf{Beschreibung}\\ 	\hline 
	15 & SCL & Clock-Signal für den \interintegratedbus-Bus\\ \hline 
	16 & SDA & Datenlinie für den \interintegratedbus-Bus\\ \hline 
	17 & DDC/CEC Ground & Einheitliche Erde\\ \hline 
	18 & +5V & Spannungsversorgung für \interintegratedbus und den \eeprom \\ \hline 	
	19 & Hot-Plug-Detect & HIGH für 'an', LOW für 'aus'\\ \hline 		
\end{tabular} 
}
\end{table}

Diese Pins müssen entsprechend Abbildung \ref{fig:hdmi-arduino} mit dem Microcontroller 
verschaltet werden\footnote{Die gesamte Pinbelegung kann in der HDMI Spezifikation nachgelesen werden (\cite[S.12]{Hdmi14Specification})}. \newpage

Die SCL- und SDA-Pins werden direkt mit den entsprechenden Pins auf dem Arduino-Board  verbunden.
Das selbe gilt für die Pins 17 und 18. Diese werden mit die GND und 5V-Pins verbunden. 
Der Hotplug-Pin kann mit einem beliebigen digitalen Ausgangspin verschaltet werden. 
Wird er auf \emph{High} gesetzt, wird das Kabel ``eingesteckt'' und analog bei \emph{Low} ``ausgesteckt''.
Hier müssen die --- von der HDMI Spezifikation vorgegebenen --- Zeiten eingehalten werden \cite[S.73]{Hdmi14Specification}.
Die dargestellten Widerstände können genutzt werden, dazu müssen die internen Pull-Up Widerstände des Arduino allerdings ausgeschaltet werden. 
Ansonsten können die Widerstände auch weggelassen werden.

\begin{figure}[H]
	\centering
	\includegraphics[width=0.5\textwidth]{hdmi-arduino}
	\caption{Verschaltung HDMI und Arduino MEGA 2560}
	\label{fig:hdmi-arduino}
\end{figure}

\subsection{Microcontroller Software}
Auf dem Microcontroller befindet sich ein Programm welches den Hot-Plug-Pin setzen kann. Über eine serielle Schnittstelle kann der Pin  geschaltet werden ohne den Versuchsaufbau 
verändern zu müssen.
Somit kann der Versuchsaufbau stehen bleiben ohne Kabel ein- und aus-stecken zu müssen. Dieses Programm kümmert sich außerdem um die Verarbeitung des \acrshort{ddc}. 
Wenn die EDID-Daten vom Host angefragt werden so sorgt es dafür, dass diese \interintegratedbus-Anfragen entsprechend beantwortet werden. Es liefert zunächst nur einen
Standard-EDID Block ohne Extensions zurück. Je nach Konfiguration können allerdings Extension-Blöcke gesendet werden. 
Des Weiteren ermöglicht es das Programm die einzelnen Bytes des 128 Byte langen EDID-Blocks frei zu verändern und korrigiert entsprechend das Prüfsummen-Byte des Blocks.

\subsection{Statische Analyse}
Zusammen mit dem Versuchsaufbau wird eine statische Analyse des Quellcodes der betroffenen Module durchgeführt. Dies ermöglicht das direkte überprüfen sowie einen umfassenden Einblick
aufgrund der Quelloffenheit von Linux.

Für den X-Server findet sich die Implementierung von \acrshort{ddc} in \path{hw/xfree86/ddc/} sowie \path{hw/xfree86/common/}\footnote{Siehe: \url{https://github.com/mirror/xserver}}.
Die entsprechende Implementierung von \acrshort{ddc} für das DRM Subsystem des Kernels liegt in \path{drivers/gpu/drm}\footnote{Siehe: \url{https://github.com/torvalds/linux}}

Durch diese Kombination wurde ein Testaufbau erzeugt welcher sowohl die konkrete Implementierung untersuchbar macht als auch die direkte Überprüfung von Vermutungen ermöglicht.
Im nächsten Abschnitt wird dabei genauer auf die Funktionsweise des Codes eingegangen. 





