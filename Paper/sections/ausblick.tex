%!TEX root = ../article.tex

% Conclusion
\section{Ausblick}
\label{sec:ausblick}
Diese Methode ist noch erweiterbar und es können weitere Verfeinerungen vorgenommen werden. Ein paar der weiteren Ansätze wären:
\begin{itemize}
\item Die Lookup-Table kann angepasst werden, um die Trennung der gefüllten von den nicht gefüllten Voxeln über andere Oberflächen vorzunehmen. Dabei ist es denkbar, zu berücksichtigen, wie sich unterschiedliche Materialien zueinander verhalten. So kann eine Methode entwickelt werden, gleichartige Materialien zu verbinden und unterschiedliche Materialien zu trennen. Dazu ist auch zu beachten, wie sich die Übergänge zwischen den Materialien an solchen Stellen verhalten.
\item Die Normalen an den Vertices können  so angepasst werden, dass ein eher fließender Übergang an den Kanten stattfindet, der die Beleuchtung beeinflusst. Dieses Verhalten könnte vom Winkel zwischen den Flächen an den Kanten abhängig gemacht werden. Es ist auch hier möglich dieses Verhalten von den Materialien abhängig zu machen, so dass es Materialien mit härteren Kanten und weicheren Kanten gibt.
\item Es ist denkbar, nicht ein Material pro Voxel zu verwenden, sondern verschiedene Mengen an verschiedenen Materialien innerhalb eines Voxels zu definieren, deren Gesamtmenge dann der Füllmenge entsprechen würde. Dazu muss die Methode zum Darstellen der Texturen auf den Voxeln angepasst werden. \cite{rvmm}
\item Den Materialien könnten noch renderspezifische Verhaltensweisen wie unterschiedliches Beleuchtungsverhalten oder Reflektionen zugewiesen werden.
\item Es gibt Methodem dazu, Normal Mapping mit Triplanarem Mapping zu verbinden. Diese Ansätze ließen sich mit den Ergebnissen dieser Arbeit vereinen. \cite{nmfts}
\item Verschiedene Methoden Schatten zu berechnen, können im Zusammenhang mit dieser Methode getestet werden.
\item Es ist denkbar zu versuchen die Methode so anzupassen, dass die Ausdehnung der Objekte im Verhältnis zu den Füllmengen der Voxel volumetrisch korrekt gerendert wird.
\end{itemize}